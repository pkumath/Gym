\documentclass[11pt]{article}
\usepackage[margin=1in]{geometry}
\usepackage{xcolor}
\usepackage{cite}
\usepackage[numbers]{natbib}
\title{Statistical Case Study}
\author{Garrett Wen}
\begin{document}
\maketitle
\tableofcontents
\section{Aug 30, 2023 at 16:30:19}

Figure out the plan and goal. 
ghggg

\begin{itemize}
	\item Factors (from data that are influential)
	\item Measures about success.
	We can give a high dimensional vector, every dimension is an indicator about some kind of success of the athletes. 
	\item Selection process (use the high dimensional )
\end{itemize}
\section{Aug 31, 2023 at 19:35:49}
Gang Wen's idea:

\subsection{Basic Info of Datasets}
  \subsection{data\_2017\_2021.csv and data\_2022\_2023.csv}
        data\_2017\_2021.csv and data\_2022\_2023.csv   files contain gymnastics performance metrics ranging from various years. The columns in this dataset are described as follows:
        \begin{itemize}
            \item LastName: The last name of the gymnast.
            \item FirstName: The first name of the gymnast.
            \item Gender: The gender of the gymnast.
            \item Country: The country that the gymnast represents.
            \item Date: The date on which the performance took place.
            \item Competition: The name of the competition.
            \item Round: The round of the competition.
            \item Location: The location where the competition took place.
            \item Apparatus: The specific gymnastics apparatus.
            \item Rank: The ranking achieved in that particular performance.
            \item D\_Score: The difficulty score assigned for the routine.
            \item E\_Score: The execution score assigned for the routine.
            \item Penalty: Any penalties incurred during the performance.
            \item Score: The total score of the performance.
        \end{itemize}

        
\subsection{Components and Their Calculation Methods}
To thoroughly assess the gymnasts, we will generate a vector with high dimensions, where each component will correspond to one of the indices listed below.:
    \begin{itemize}
        \item \textbf{Overall Score}: This is computed as the average `Score' across all performances for each gymnast.
        \item \textbf{Average D-Score}: This is calculated as the average `D\_Score' for each gymnast across all performances.
        \item \textbf{Average E-Score}: This is calculated as the average `E\_Score' for each gymnast across all performances.
        \item \textbf{Average Penalty}: This is the average `Penalty' incurred by each gymnast across all performances.
        \item \textbf{Specialty Score}: This is the average `Score' on the apparatus where each gymnast has the highest average performance.
        \item \textbf{Consistency Score}: This is calculated as the standard deviation of the `Score' for each gymnast across all performances.
    \end{itemize}
\textcolor{red}{Note that when calculating those `averages', we may weight them according to time, i.e. more recent time score will get more weights.}
\subsection{Find a Metric to Define `Success'}
    To synthesize these metrics into a singular measure of each gymnast's potential for future performance, we need a function to map this vector to a final metric. For example, a linear weighted model will be used. The Composite Score for each gymnast will be calculated using the following formula:
    \[
    \textit{Composite Score} = w_1 \times \textit{Overall Score} + w_2 \times \textit{Average D-Score} + w_3 \times \textit{Average E-Score}\]\[ - w_4 \times \textit{Average Penalty} + w_5 \times \textit{Specialty Score} - w_6 \times \textit{Consistency Score}
    \]
    Here, \( w_1 \) through \( w_6 \) are the weights assigned to each metric according to possible preferences. The gymnast with the highest Composite Score is deemed to be the most promising for future competitions. We can also use more complex models for the final metric, possibly a non-linear method by machine learning.
    \subsection{Final Remarks}
    \textcolor{red}{Finally, it should be noted that we should certainly give more weights to the data for recent times (that means, we should try to weight those scores according to time). Additionally, we should also strive to find more useful data and conduct similar analyses.}
    \section{Sep 13, 2023 at 14:07:37}
    \subsection{Produce Composite Scores and Predictions}
    
\end{document}
\bibliographystyle{unsrtnat}
\bibliography{StatisticalCaseStudy.bib}
\end{document}
