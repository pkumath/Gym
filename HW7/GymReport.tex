\documentclass{article}
\usepackage[margin=1in]{geometry}
\usepackage{cite}
\usepackage{mathbbol}
\usepackage{xcolor}
\usepackage[colorinlistoftodos]{todonotes}
\usepackage{amsmath}
\usepackage{hyperref}
\usepackage{float}
\usepackage{tikz}
\usetikzlibrary{positioning, fit, arrows.meta, shapes}

% used to avoid putting the same thing several times...
% Command \empt{var1}{var2}
\newcommand{\empt}[2]{$#1^{\langle #2 \rangle}$}
\usepackage{booktabs} 
\usepackage{amsthm}
\usepackage{amssymb}
\usepackage{multirow}
% Define a new theorem style
\newtheoremstyle{roman}  % name
  {}                     % Space above
  {}                     % Space below
  {\rmfamily}            % Body font
  {}                     % Indent amount
  {\bfseries}            % Theorem head font
  {.}                    % Punctuation after theorem head
  {.5em}                 % Space after theorem head
  {}                     % Theorem head spec (can be left empty, meaning ‘normal’)

% Now we use the newly defined style
\theoremstyle{roman}
\newtheorem{proposition}{Proposition}
\newtheorem{lemma}{Lemma}
\newtheorem{definition}{Definition}
\newtheorem{theorem}{Theorem}
\newtheorem{corollary}{Corollary}
\newtheorem{assumption}{Assumption}
\usepackage[numbers]{natbib}
\title{Gymnastic Report}
\author{Garrett Wen}
\begin{document}
\maketitle
\tableofcontents
\section{Rules Reminder}
\subsection{Apparatus}
women(6): team all-around, individual all-around, vault, uneven bars, balance beam, and floor exercise
\\
men (8): team all-around, individual all-around, floor exercise, pommel horse, still rings, vault, parallel bars, and high bar

Specifically,

For women: (4 apparatus)

$\mathrm{BB}=$ Balanced Beam

$\mathrm{VT}=$ Vault

$\mathrm{FX}=$ Floor Exercise

$\mathrm{UB}=$ Uneven Bars

For men: (6 apparatus)

$\mathrm{VT}=$ Vault

$\mathrm{SR}=($ Still) Rings

$\mathrm{PH}=$ Pommel Horse

$\mathrm{PB}=$ Parallel Bars

$\mathrm{HB}=$ Horizontal Bars

$\mathrm{FX}=$ Floor Exercise

Note: VT indicates that only 1 vault was performed VT1 may indicate that only 1 vault was performed OR it could indicate the 1 st of 2 vaults that were performed

\subsection{Initial Setting}
\begin{center}
\begin{tabular}{|l|l|l|}
\hline
Category & Men & Women \\
\hline
Total Athletes & 96 & 96 \\
\hline
Number of Teams & 12 & 12 \\
\hline
Athletes per Team & 5 & 5 \\
\hline
Total Athletes in Team Events & 60 (12 teams x 5 athletes) & 60 (12 teams x 5 athletes) \\
\hline
\begin{tabular}{l}
Individual Athletes (from \\
countries without full teams) \\
\end{tabular} & 36 (Max 3 per country) & 36 (Max 3 per country) \\
\hline
\end{tabular}
\end{center}


\subsection{Rules Summary}
\begin{center}
\begin{tabular}{|l|l|}
\hline
Phase & Rule \\
\hline
Team Composition & \begin{tabular}{l}
Each team consists of 5 athletes, but only 4 \\
compete on each apparatus during the qualify- \\
ing round. \\
\end{tabular} \\
\hline
Qualifying Round (4 up, 3 count) & \begin{tabular}{l}
4 out of the 5 athletes on each team compete \\
on each apparatus. The top 3 scores on each \\
apparatus are summed for team placement. \\
\end{tabular} \\
\hline
Team Finals (3 up, 3 count) & \begin{tabular}{l}
Top 8 teams from the qualifying round com- \\
pete. Scores are based on the "3 up, 3 count" \\
rule, meaning all 3 athletes' scores on each ap- \\
paratus count towards the team total. \\
\end{tabular} \\
\hline
Individual All-Around Finals & \begin{tabular}{l}
Athletes must compete on all apparatuses in \\
the qualifying round to be eligible. The top 24 \\
athletes qualify, with a max of 2 gymnasts per \\
country. \\
\end{tabular} \\
\hline
Apparatus Finals & \begin{tabular}{l}
The top 8 athletes on each apparatus from the \\
qualifying round advance, with a maximum of \\
2 gymnasts per country. \\
\end{tabular} \\
\hline
\end{tabular}
\end{center}


\subsection{Note: Qualifying Round}
Individual Qualification: Countries that do not have a full team can qualify a maximum of 3 individual athletes per gender. These individual gymnasts are allowed to compete in the qualifying round to get into apparatus final.

Apparatus Participation: The individual gymnasts representing countries without a full team can participate on all apparatuses in the qualifying round. This allows them to be considered for both the individual all-around and apparatus finals.

\subsection{Rules by Event}
\subsubsection{Team All-Around Event}
\begin{center}
\begin{tabular}{|l|l|}
\hline
Event & Rule \\
\hline
Qualifying Round & - Each team has 5 athletes. \\
 & - 4 out of the 5 athletes compete on each \\
 & apparatus. \\
 & - Top 3 scores on each apparatus are \\
 & summed for the team's total. \\
 & - Top 8 teams based on this total advance \\
 & to the final. \\
\hline
Team All-Around Final & - Qualifying scores are discarded; teams \\
 & start fresh. \\
 & - Each team can use any of its 5 athletes \\
 & on each apparatus. \\
 & - Scores of 3 athletes on each apparatus \\
 & count (" 3 up, 3 count"). \\
 & - Team with the highest total across all apparatuses wins. \\ \hline
\end{tabular}
\end{center}



\subsubsection{Individual All-Around Event}
\begin{center}
\begin{tabular}{|l|l|}
\hline
Phase & Rule \\
\hline
Qualifying & - Athletes must compete on all appara- \\
 & tuses in the qualifying round to be eli- \\
 & gible. \\
 & - The top 24 athletes based on their com- \\
 & bined scores across all apparatuses ad- \\
 & vance. \\
 & - Maximum of two gymnasts per country \\
 & can qualify for the individual all-around \\
 & final. \\
\hline
Final & - Qualifying scores are discarded; gym- \\
 & nasts start fresh. \\
 & - Gymnasts compete on all apparatuses. \\
 & - Combined scores across all apparatuses \\
 & determine final placements. \\
 & - The gymnast with the highest total score
is the winner. \\\hline
\end{tabular}
\end{center}


\subsubsection{Each Apparatus Event}
\begin{center}
\begin{tabular}{|l|l|}
\hline
Phase & Rule \\
\hline
Qualifying & - Athletes compete on their chosen appa- \\
 & ratuses during the qualifying round. \\
 & - The top 8 athletes on each apparatus ad- \\
 & vance to the final for that apparatus. \\
 & - Maximum of two gymnasts per country \\
 & can qualify for each apparatus final. \\
\hline
Final & - Qualifying scores for the apparatus are \\
 & discarded; gymnasts start fresh. \\
 & - Gymnasts compete on the specific apparatus for which they qualified. \\
 & - The athlete with the highest score on that \\
 & apparatus is the winner. \\
\hline
\end{tabular}
\end{center}

\section{Data Organization}
\subsection{Basis Classes and Objects}
\subsection{Functions Explained}
\section{Predicting Model}
\subsection{Countries and Teams}
\subsection{Gymnasts}
\section{Result}

\section*{Remark}
We would like to express that translating Python code to R code is indeed a challenging task. We sincerely apologize for the projects written in Python; however, we would like to mention that we have dedicated several days to translating them. We kindly request future students in this course to refrain from attempting this, as it will ultimately result in a significant waste of time.
\bibliographystyle{unsrtnat}
\bibliography{GymReport.bib}
\end{document}
